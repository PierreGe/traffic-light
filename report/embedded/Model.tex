\section{Model}

\subsection{Introduction}
To build our final Uppaal model, we decided to use an incremental way for the construction, meaning that we started from a Minimum Viable Product (MVP) that we develop a little bit more at each step. We verified each one and, that way, we could assure ourselves to have a solid base for the next step. Thanks to this approach, we could identify possible issues more easily.

We finally did it in 3 different steps and we will now explain each of them.

\subsubsection{Step 1: Basic Traffic Lights system}
In this first basic step, we created an Uppaal model of a simple controller sending a pulse every 3 seconds to the two direction traffic lights. There is no pedestrians and no buses in this step. \\
Every 3 seconds, the North-South traffic lights switch from green to red while the East-West traffic lights switch, on the opposite, from red to green. \\

Thanks to this basic example, we understood Uppaal basics and notably how it is possible to send messages from one component to an other component thanks to the channel message passing system. \\

\begin{figure}[H]\label{fig:step1}
  \centering
    \includegraphics[width=0.5\textwidth]{picture/model/trafficlight_step1_s1.png}
    \caption{Model of the step 1}
\end{figure}

\subsubsection{Step 2: Pedestrians}
In this step, pedestrians were added and different things had to be changed to accommodate this new change. \\ 

When a pedestrians call was send by the environment, the car's traffic light that was green had to become red and the other car traffic light had to stay red. When those two traffic lights are red, the pedestrians traffic light could switch to green. Because we also wanted fairness between the different traffic lights in our system, two assumptions were added here:
\begin{enumerate}
    \item When the pedestrians traffic light is green, we have to remember which car's traffic light was previously green. This will be useful for the next assumption.
    \item We wanted fairness between the actors, even if pedestrians have the priority. The problem at the beginning was that starvation could happen if, when the pedestrians lights are green, we always switched to a specific car's traffic light. This would mean that the other car's traffic light would remain red if pedestrians are always calling. We thus added what we called a "delayed" call where, even if the pedestrians are always pressing the button, the two cars lights have to be green at least once before the pedestrians could have their lights green again. This is why the first assumption is useful, we have to remember which car's light was previously green so the cars in the other side of the crossroad don't always have to wait 2 times to be green again. \\
    As an example, if the East-West car's traffic light is green and a pedestrian pressed the button, after the pedestrians lights switch from green to red, the North-South car's traffic light is switched to green so they do not have to wait another time.
\end{enumerate}
\textbf{Fairness} in our project means that all traffic lights should be green for approximately the same amount of time.

\subsubsection{Step 3: Buses}\label{sec:step3}
In this step, we added the buses generated by the environment. The idea here is that the buses have priority on the rest of the traffic. \\
When a bus is generated, the next traffic light that has to be green is the one of the bus. When the bus is crossing, all other lights are red because the bus will go through the entire crossroad. The model can be seen as on Figure \ref{fig:step3bus}. \\

\begin{figure}[H]\label{fig:step3bus}
  \centering
    \includegraphics[width=0.5\textwidth]{picture/model/trafficlight_step3_s2.png}
    \caption{Model for the step 3}
\end{figure}


\noindent Even if the pedestrians have called to have the green lights before the bus, if a bus is generated between the call and the pedestrians green lights, we will first give the green light to the bus, then the pedestrians, then the cars. This mean that the priority hierarchy is \textit{Buses-Pedestrians-Cars}.


\subsection{Uppaal Timed Automatons}
In order to formally model the system, we use several timed automatons from Uppaal. While some of those automatons are actors in the environment, the others are part of the controller.
The theory tells us that, when the environment is playing, we, humans, cannot decide the action that will be taken. For example, buses and pedestrians arrivals are unpredictable and uncontrollable, so they belong to the environment.
The actors which implement the solution for avoiding crashes belong to the controller. \\

\subsubsection{Environment}
\paragraph{Crosswalk} \mbox{}\\
We will here talk about the mechanisms of our crosswalk generator done in our step 3 from Section \ref{sec:step3}. \\

\begin{figure}[H]\label{fig:crosswalk}
  \centering
    \includegraphics[width=0.9\textwidth]{picture/crosswalk.png}
    \caption{Pedestrians generator}
\end{figure}
\noindent The initial state of the pedestrians lights is Red. From this state, three different destinations states are possible:
\begin{enumerate}
  \item The Preempted state: A pedestrian call has been made but there is a bus. Since they have total priority, we will first have to wait for the bus to cross the road before we can,
  \item The DelayedCall state: If there is no bus but the pedestrians already pushed the button within a certain time period in the past, we do not give them the green light immediately to stay fair,
  \item The Called state: If a call has been made and no conditions from 1 or 2 are present, we go to this state. This state means that the pedestrians will get the green light at the end of the current timer.
\end{enumerate}
At the end, every other states will reach the Preempted one after having to wait for the end of the timer. We then switch the pedestrians lights to green. After the green state, we reach a free state because, before going back to the red state, different things can happen. Indeed, a bus could arrive and since it has the priority, we will have to give him the next green light. If no bus arrived, we go back to the red state and the basic cars traffic lights will start again. 

\paragraph{Bus} \mbox{}\\
We will here talk about the mechanisms of our bus generator done in our step 3 from Section \ref{sec:step3}. \\
As for the pedestrians, different states can be reached. We still have Preempted, DelayedCall and Called and their meaning is the same than the ones previously defined for the pedestrians. The only difference here is that, since the bus has priority, if the pedestrians were waiting, we first have to put the bus light to green. This means that there is 4 different states that can be reached while there are only 3 reachable states for the pedestrians. The rest follows the same scheme as for pedestrians lights.

\begin{figure}[H]\label{fig:bus}
  \centering
    \includegraphics[width=0.9\textwidth]{picture/bus.png}
    \caption{Buses generator}
\end{figure}

\noindent Those were the two actors controlled by the environment.
\newpage
\subsubsection{Controller}
\paragraph{Traffic Lights} \mbox{}\\
We will here only look at the West-East traffic light since the South-North lights follow the same idea but the Green and Red states are inverted.
\begin{figure}[H]\label{fig:crosswalk}
  \centering
    \includegraphics[width=0.9\textwidth]{picture/leftright.png}
    \caption{West-East traffic lights}
\end{figure}



From the Red state, three different states can be reached:
\begin{enumerate}
  \item If a bus has been called, other traffic lights have to remain red so the bus can drive through the crosswalk. We thus reach a \textit{Bus} state.
  \item If pedestrians have pushed the button, we switch to a \textit{Crosswalk} state.
  \item If no buses or pedestrians were generated and the other cars traffic lights are switching to red, we can reach a \textit{Green} state.
\end{enumerate}
From the Crosswalk and Bus state, we still have to check, before going back to a Green or Red state if no bus or pedestrian call has been made. If it is the case, we have to take the correct transition by checking the Boolean values \textit{pedc} and \textit{busc}.
